\section{Vorwort}
Die Wurzel von KFlog reicht bis in das Jahr 1998 zur�ck.
Das Projekt wurde von einem Geographie-Studenten und einem Segelflieger ins Leben gerufenen und war urspr�nglich dazu gedacht, ein bisschen Programmieren zu lernen ...
Damit ist KFlog eine nicht ganz neue, jedoch noch recht unbekannte Software
f�r Segelflieger.
\\
Der Hauptunterschied zu anderen Programmen wie z.B. StrePla besteht darin, dass KFlog eine Anwendung f�r alle Plattformen, auf denen die KDE-Umgebung l�uft, ist (z.B. Linux oder Solaris) und das KFlog unter der GNU PUBLIC LICENSE (Open Source) steht und somit frei verf�gbar ist.
\\
Die Entwicklung des Programms erfolgt durch ein engagiertes Team, das �ber den ganzen Globus verteilt ist und via Internet miteinander in Verbindung steht.
Zur Zeit arbeiten etwa 20 Leute an der Entwicklung von KFlog. Die meisten Mitglieder des Teams sind mit der ''Materie des Segelfliegens'' vertraut so dass man fast behaupten kann, dass KFlog von Fliegern f�r Flieger geschrieben wurde. Das bringt den Vorteil mit, dass gewisse Bed�rfnisse des  Marktes schon bei der Entwicklung mit ber�cksichtigt werden. \\

KFlog ist z.Z. in sechs Sprachen verf�gbar und wird voraussichtlich noch in weitere Sprachen �bersetzt werden.\\

Die k�nftige Erweiterung der Hardwarekompertibilit�t zu verschiedenen Ger�ten ist davon abh�ngig, wie die einzelnen Hersteller unser Team mit den notwendigen Informationen zur Programmierung der Schnittstelen der Logger versorgen.


